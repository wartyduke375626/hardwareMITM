\chapter*{Úvod} % chapter* je necislovana kapitola
\addcontentsline{toc}{chapter}{Úvod} % rucne pridanie do obsahu
\markboth{Úvod}{Úvod} % vyriesenie hlaviciek

Mnohé hardvérové zariadenia pozostávajú z viacerých integrovaných obvodov a riadiacich jednotiek prepojených komunikačnými zbernicami. V rámci komunikácie po zberniciach si jednotlivé komponenty často vymieňajú rôzne citlivé dáta ako napríklad rôzne konfiguračné parametre alebo šifrovacie kľúče.  Z bezpečnostného hľadiska je preto dôležité zabezpečiť integritu, autentickosť a dôvernosť informácie prenášanej po zberniciach.

Keďže útoky na hardvérové zbernice si vyžadujú fyzický prístup k zariadeniu, bezpečnosť tejto komunikácie sa pri návrhu často zanedbáva. Medzi dnešné hardvérové zariadenia však patria aj rôzne koncové zariadenia, napríklad smartfóny, rôzne systémy inteligentných domácností, bezpečnostné zámky, senzory a pod. Pri takýchto zariadeniach je často fyzický prístup útočníka realistický predpoklad a zanedbaná bezpečnosť komunikácie po zberniciach môže predstavovať relevantnú hrozbu.

Jedným z možných zbernicových útokov je aktívne zasahovanie do komunikácie medzi týmito obvodmi, napríklad za účelom zmeny konfiguračných parametrov, získania kľúčov alebo modifikácie odpovede. Takýmito hardvérovými „Man-in-the-middle“ (MITM) útokmi sa zaoberalo viacero prác \cite{mitmCAN, mitmI2C, mitmSmartphone, mitmBitlocker, mitmTouch, mitmTPM}. Tieto útoky zneužívajú práve nedostatočnú ochranu komunikácie medzi hardvérovými komponentami.

V tejto práci implementujeme hardvérový MITM útok pre komunikačné zbernice UART a SPI. Útok bude implementovaný pomocou FPGA, čo umožňuje zasahovať do komunikácie v reálnom čase. Dôraz kladieme na všeobecný návrh FPGA obvodu čo umožňuje obvod ľahko upraviť pre účely implementácie a analýzy nových útokov na rôzne zbernice a aplikácie. Cieľom preto bude oddeliť zbernicové rozhranie od logiky MITM útoku, čo umožní čiastočne abstrahovať od detailov zbernice ako napríklad fyzické kódovanie signálov. V kapitole \ref{kap:implementacia} poukážeme na niektoré problémy spojené s~asymetrickou master-slave architektúrou, od ktorej principiálne nemožno úplne abstrahovať. Konfigurácia FPGA obvodu bude prispôsobiteľná parametrami a syntéza automatizovaná skriptom.

V kapitole \ref{kap:principy} predstavíme základné princípy hardvérových MITM útokov a ich odlišnosti od bežných MITM útokov uvažovaných v scenári sietí väčšieho rozsahu (typu LAN alebo internet). V kapitole \ref{kap:zbernice} predstavíme niekoľko zbernicových protokolov a ich vlastnosti, čo umožní lepšie pochopenie implementácie FPGA obvodu. Hlavnou časťou práce je kapitola \ref{kap:implementacia}, v ktorej detailne predstavíme náš návrh a implementáciu jednotlivých častí FPGA obvodu, pričom budeme brať do úvahy aj štandardné problémy pri návrhu FPGA obvodov (napr. synchronizácia vstupov, odstránenie zákmitov signálu). Na záver, v kapitole \ref{kap:priklady} demonštrujeme funkčnosť a spôsob použitia nášho riešenia na dvoch príkladoch zásahov do komunikácie v jednoduchých scenároch.