\chapter*{Záver}  % chapter* je necislovana kapitola
\addcontentsline{toc}{chapter}{Záver} % rucne pridanie do obsahu
\markboth{Záver}{Záver} % vyriesenie hlaviciek

V práci sa nám podarilo implementovať hardvérový MITM útok pre komunikačné zbernice pomocou FPGA obvodu. Obvod sa podarilo navrhnúť dostatočne všeobecne a bolo možné do veľkej mieri oddeliť logiku MITM útoku od zbernicového rozhrania. Zároveň sme poukázali na niektoré vlastnosti zbernicových protokolov, ktoré principiálne nemožno pred MITM logikou abstrahovať. Príkladom takej vlastnosti je asymetrická master-slave architektúra, ktorá prináša obmedzenia. Keďže riadenie komunikácie pri takýchto zberniciach môže závisieť aj od obsahu samotných dát, nie je možné od neho úplne abstrahovať.

V rámci zbernicového rozhrania sme implementovali ovládače pre komunikačné zbernice UART a SPI, ktoré zohľadňujú ich odlišné vlastnosti, ako asynchrónnosť UART a synchrónnosť SPI zbernice. Implementácia celého FPGA obvodu je zároveň rozdelená do viacerých parametrizovaných modulov, ktoré umožňujú flexibilnú úpravu FPGA obvodu podľa potreby. Pre tento účel sme v častiach \ref{sek:primitives} a \ref{sek:io} navrhli aj niekoľko základných stavebných prvkov (primitív) -- detektor hrán signálu, buffer, čítač, generátor impulzov, synchronizátor a ďalšie, ktoré je možné znovu použiť pri rozširovaní implementácie o~dodatočnú funkcionalitu.

Pri implementácii sme dbali aj na štandardné problémy pri návrhu FPGA obvodov, ako napríklad synchronizácia vstupov a odstránenie zákmitov signálu (angl. signal debouncing). Konfigurácia obvodu je prispôsobiteľná cez parametre, ktoré sme opísali v časti \ref{sek:config} a syntéza je automatizovaná skriptom. Použiteľnosť nášho riešenia sme na záver demonštrovali na dvoch konkrétnych príkladoch útoku na implementované zbernice UART a SPI. V rámci týchto príkladov sme zároveň predstavili ukážku návrhu MITM logiky pre náš obvod.

Za hlavný výsledok práce považujeme práve všeobecný návrh FPGA obvodu, ktorý umožňuje rozširovať FPGA obvod o dodatočnú funkcionalitu a jednoduchú úpravu pre rôzne zbernicové útoky. V práci sme tým poukázali na aplikovateľnosť princípov softvérového inžinierstva na návrh FPGA obvodov. Práca demonštruje aj výhody FPGA oproti softvérovým riešeniam, ako je rýchlosť a kontrola spracovania informácie na úrovni logických hradiel, a ponúka rozšíriteľný základ pre ďalšie zbernicové protokoly.

Priamim nadviazaním na prácu môže byť rozšírenie vytvoreného základného riešenia. Špeciálne zaujímavým rozšírením by bola podpora vysoko-rýchlostných zberníc (napr. PCIe, USB 3.0), ktoré si vyžadujú špecializovaný hardvér pre prijímanie a vysielanie rámcov. Ďalším nadviazaním je využitie FPGA MITM obvodu na analýzu komplexnejších zbernicových útokov.
