\documentclass[12pt, twoside]{book}
%\documentclass[12pt, oneside]{book}  % jednostranna tlac

%spravne nastavenie okrajov
\usepackage[a4paper,top=2.5cm,bottom=2.5cm,left=3.5cm,right=2cm]{geometry}
%zapnutie fontov pre UTF8 kodovanie
\usepackage[utf8]{inputenc}
\usepackage[T1]{fontenc}

%zapnutie slovenskeho delenia slov
%a automatickych nadpisov ako Obsah, Obrázok a pod. v slovencine
\usepackage[slovak]{babel} % vypnite pre prace v anglictine!

%nastavenie riadkovania podla smernice
\linespread{1.25} % hodnota 1.25 by mala zodpovedat 1.5 riadkovaniu

% balicek na vkladanie zdrojoveho kodu
\usepackage{listings}

% nastavenia balicka listings
\renewcommand{\lstlistingname}{Úryvok}
\lstset{extendedchars=true, basicstyle=\footnotesize, frame=lines, tabsize=4,
    commentstyle=\color{olive}\textit, keywordstyle=[1]\color{blue},
    literate=
    {á}{{\'a}}1 {ä}{{\"a}}1 {č}{{\v{c}}}1 {ď}{{\v{d}}}1 {é}{{\'e}}1 {í}{{\'i}}1
    {ĺ}{{\'l}}1 {ľ}{{\v{l}}}1 {ň}{{\v{n}}}1 {ó}{{\'o}}1 {ô}{{\^o}}1 {š}{{\v{s}}}1
    {ť}{{\v{t}}}1 {ú}{{\'u}}1 {ý}{{\'y}}1 {ž}{{\v{z}}}1
    {Á}{{\'A}}1 {Č}{{\v{C}}}1 {Ď}{{\v{D}}}1 {É}{{\'E}}1 {Í}{{\'I}}1 {Ĺ}{{\'L}}1 
    {Ľ}{{\v{L}}}1 {Ň}{{\v{N}}}1 {Ó}{{\'O}}1 {Š}{{\v{S}}}1 {Ť}{{\v{T}}}1 {Ú}{{\'U}}1
    {Ý}{{\'Y}}1 {Ž}{{\v{Z}}}1
    {~}{{$\sim$}}1 {-}{{\textendash}}1,
}
\lstdefinelanguage{AVR}{
    keywords={clr, eor, cpi, breq, inc, tst, brne, ldi, sbi, cbi, nop, add, sub, mov, or, dec, ldr, str},
    sensitive=false,
    comment=[l]{;},
}

% balicek na vkladanie obrazkov
\usepackage{graphicx}
\usepackage{subfig}
% balicek pre pouzitie hrubych ciar
\usepackage{boldline}
% balicek pre viacriadkove tabulky
\usepackage{multirow}
% balicek na vkladanie celych pdf dokumentov
\usepackage{pdfpages}
% balicek na vkladanie diagramov
\usepackage{pgfplots}
\pgfplotsset{width=0.6\textwidth,compat=1.9}
% balicek na spravne formatovanie URL
\usepackage{url}
% balicek na hyperlinky v ramci dokumentu
% zrusime farebne ramiky okolo liniek
\usepackage[hidelinks,breaklinks]{hyperref}



% -------------------
% --- Definicia zakladnych pojmov
% --- Vyplnte podla vasho zadania, rok ma byt rok odovzdania
% -------------------
\def\mfrok{2025}
\def\mfnazov{Hardvérové MITM útoky\\na komunikáciu po zberniciach}
\def\mftyp{Diplomová práca}
\def\mfautor{Bc. Dennis Vita}
\def\mfskolitel{RNDr. Richard Ostertág, PhD. }

\def\mfmiesto{Bratislava, \mfrok}

\def\mfodbor{ Informatika}
\def\program{ Informatika }
\def\mfpracovisko{ FMFI.KI - Katedra informatiky }

\begin{document}
\frontmatter
\pagestyle{empty}

% -------------------
% --- Obalka ------
% -------------------
\begin{center}
\sc\large
Univerzita Komenského v Bratislave\\
Fakulta matematiky, fyziky a informatiky

\vfill

{\LARGE\mfnazov}\\
\mftyp
\end{center}

\vfill

{\sc\large 
\noindent \mfrok\\
\mfautor
}

\cleardoublepage
% --- koniec obalky ----


% -------------------
% --- Titulný list
% -------------------
\noindent
\setcounter{page}{1}

\begin{center}
\sc  
\large
Univerzita Komenského v Bratislave\\
Fakulta matematiky, fyziky a informatiky

\vfill

{\LARGE\mfnazov}\\
\mftyp
\end{center}

\vfill

\noindent
\begin{tabular}{ll}
Študijný program: & \program \\
Študijný odbor: & \mfodbor \\
Školiace pracovisko: & \mfpracovisko \\
Školiteľ: & \mfskolitel \\
\end{tabular}

\vfill


\noindent \mfmiesto\\
\mfautor

\cleardoublepage
% --- Koniec titulnej strany


% -------------------
% --- Zadanie z AIS
% -------------------
\newpage 
\includepdf{images/zadanie.pdf}

\cleardoublepage
% --- Koniec zadania


% -------------------
%   Poďakovanie - nepovinné
% -------------------
\newpage
\pagestyle{plain}
~

\vfill
{\bf Poďakovanie:} Ďakujem vedúcemu diplomovej práce RNDr. Richardovi Ostertágovi, PhD. za odborné vedenie, metodickú pomoc, podnetné nápady, pripomienky a~konzultácie pri písaní práce. Ďakujem aj Katedre informatiky a Fakulte matematiky, fyziky a informatiky za kvalitné vzdelanie, veľkú ochotu pomáhať počas štúdia a~cenné vedomosti, ktoré okrem iného prispeli aj k vypracovaniu tejto práce.

Táto práca vznikla aj vďaka podpore v rámci Operačného programu Integrovaná infraštruktúra pre projekt: Advancing University Capacity and Competence in Research, Development and Innovation (ACCORD, ITMS2014+:313021X329), spolufinancovaný zo zdrojov Európskeho fondu regionálneho rozvoja.
% --- Koniec poďakovania


% -------------------
%   Abstrakt - Slovensky
% -------------------
\newpage 
\section*{Abstrakt}

Mnohé dnešné hardvérové zariadenia pozostávajú z viacerých integrovaných obvodov, ktoré sú navzájom prepojené komunikačnými zbernicami. Zaujímavým aspektom je aktívne zasahovanie do takejto komunikácie, napríklad za účelom modifikácie údajov alebo získania kľúčov. V práci sme implementovali hardvérový \uv{Man-in-the-middle} (MITM) útok pre zbernice UART a SPI pomocou FPGA, čo umožňuje do komunikácie zasahovať v reálnom čase. Návrh oddeľuje zbernicové rozhranie od MITM logiky, čo zjednodušuje prispôsobenie útoku na iné zbernice. Implementácia je modulárna a rozšíriteľná, s opakovane použiteľnými základnými stavebnými prvkami (napr. detektor hrán signálu, buffer, čítač a pod.). V rámci zbernicového rozhrania sú pre UART a SPI zbernice vytvorené špecifické ovládače, ktoré zohľadňujú ich odlišné vlastnosti (asynchrónnosť UART a synchrónnosť SPI). Zároveň poukazujeme na principiálne neabstrahovateľné vlastnosti zbernice ako napríklad master-slave architektúra. Tá prináša obmedzenia, ktoré môžu závisieť od obsahu prenášaných dát. Pri implementácii sme dbali aj na štandardné problémy pri návrhu FPGA obvodov ako napríklad synchronizácia vstupov a odstránenie zákmitov signálu (angl. signal debouncing). Konfigurácia obvodu je prispôsobiteľná cez parametre (napr. frekvencia systémových hodín) a syntéza je automatizovaná skriptom. Práca demonštruje výhody FPGA oproti softvérovým riešeniam, ako je rýchlosť a kontrola spracovania informácie na úrovni logických hradiel, a ponúka rozšíriteľný základ pre ďalšie zbernicové protokoly. Použiteľnosť riešenia na záver demonštrujeme na dvoch konkrétnych príkladoch útokov.

\paragraph*{Kľúčové slová:} MITM útok, zbernica, FPGA
% --- Koniec Abstrakt - Slovensky


% -------------------
% --- Abstrakt - Anglicky 
% -------------------
\newpage 
\section*{Abstract}

Many of today’s hardware devices consist of multiple integrated circuits interconnected by communication buses. An interesting aspect is to actively interfere with such communication, for example, to modify data or extract keys. In this thesis, we implemented a hardware `Man-in-the-Middle' (MITM) attack for UART and SPI buses using an FPGA, which enables real-time communication intervention. The design separates the bus interface from the MITM logic, which simplifies the adaptation of the attack for other bus types. The implementation is modular and extensible, with reusable basic building blocks (e.g., signal edge detector, buffer, counter, etc.). Within the bus interface, specific controllers have been developed for the UART and SPI buses, addressing their different characteristics (UART’s asynchronicity versus SPI’s synchronicity). We also highlight fundamentally non-abstractable bus properties such as the master-slave architecture, which introduces constraints that may depend on the content of the transmitted data. In our implementation, we also addressed common FPGA design problems such as input synchronization and signal debouncing. The FPGA configuration is customizable via parameters (e.g., system clock frequency) and synthesis is automated through a script. Our work demonstrates the advantages of an FPGA over software solutions, such as processing speed and fine-grained control at the level of logic gates, and provides an extensible foundation for additional bus protocols. Finally, we demonstrate the applicability of our solution on two concrete examples of attacks.

\paragraph*{Keywords:} MITM attack, bus, FPGA
% --- Koniec Abstrakt - Anglicky


% -------------------
% --- Predhovor - v informatike sa zvacsa nepouziva
% -------------------
%\newpage 
%
%\chapter*{Predhovor}
%
%Predhovor je všeobecná informácia o práci, obsahuje hlavnú charakteristiku práce 
%a okolnosti jej vzniku. Autor zdôvodní výber témy, stručne informuje o cieľoch 
%a význame práce, spomenie domáci a zahraničný kontext, komu je práca určená, 
%použité metódy, stav poznania; autor stručne charakterizuje svoj prístup a svoje 
%hľadisko. 
%
% --- Koniec Predhovor


% -------------------
% --- Obsah
% -------------------
\newpage 

\tableofcontents
% ---  Koniec Obsahu


% -------------------
% --- Zoznamy tabuliek, obrázkov - nepovinne
% -------------------
\newpage 

\listoffigures
\listoftables
% ---  Koniec Zoznamov


\mainmatter
\pagestyle{headings}


\input 00-uvod.tex 

\input 01-principy.tex

\input 02-zbernice.tex

\input 03-implementacia.tex

\input 04-priklady.tex

\input 05-zaver.tex


% -------------------
% --- Bibliografia
% -------------------
\newpage	

\backmatter

\thispagestyle{empty}
\clearpage

\bibliographystyle{plain}
\bibliography{literatura} 
%---koniec Referencii


% -------------------
%--- Prilohy---
% -------------------

\input 06-prilohaCD.tex

\end{document}